% Copyright 2004 by Till Tantau <tantau@users.sourceforge.net>.
%
% In principle, this file can be redistributed and/or modified under
% the terms of the GNU Public License, version 2.
%
% However, this file is supposed to be a template to be modified
% for your own needs. For this reason, if you use this file as a
% template and not specifically distribute it as part of a another
% package/program, I grant the extra permission to freely copy and
% modify this file as you see fit and even to delete this copyright
% notice. 

\documentclass{beamer}
% There are many different themes available for Beamer. A comprehensive
% list with examples is given here:
% http://deic.uab.es/~iblanes/beamer_gallery/index_by_theme.html
% You can uncomment the themes below if you would like to use a different
% one:
%\usetheme{AnnArbor}
%\usetheme{Antibes}
%\usetheme{Bergen}
%\usetheme{Berkeley}
%\usetheme{Berlin}
%\usetheme{Boadilla}
%\usetheme{boxes}
%\usetheme{CambridgeUS}
%\usetheme{Copenhagen}
%\usetheme{Darmstadt}
%\usetheme{default}
%\usetheme{Frankfurt}
%\usetheme{Goettingen}
%\usetheme{Hannover}
%\usetheme{Ilmenau}
%\usetheme{JuanLesPins}
%\usetheme{Luebeck}
\usetheme{Madrid}
%\usetheme{Malmoe}
%\usetheme{Marburg}
%\usetheme{Montpellier}
%\usetheme{PaloAlto}
%\usetheme{Pittsburgh}
%\usetheme{Rochester}
%\usetheme{Singapore}
%\usetheme{Szeged}
%\usetheme{Warsaw}

\usepackage{multimedia}

%\usebackgroundtemplate{\includegraphics[width= \paperwidth, height=\paperheight]{imagen2}}

\title{Sistema de control autónomo para robot en FPGAs libres}

% A subtitle is optional and this may be deleted
\subtitle{}

\author{Juan Ordóñez Cerezo\inst{1}}
% - Give the names in the same order as the appear in the paper.
% - Use the \inst{?} command only if the authors have different
%   affiliation.

\institute[Universidad de Granada] % (optional, but mostly needed)
{
  \inst{1}%
  Universidad de Granada
}
% - Use the \inst command only if there are several affiliations.
% - Keep it simple, no one is interested in your street address.

\date{}
% - Either use conference name or its abbreviation.
% - Not really informative to the audience, more for people (including
%   yourself) who are reading the slides online

\subject{}
% This is only inserted into the PDF information catalog. Can be left
% out. 

% If you have a file called "university-logo-filename.xxx", where xxx
% is a graphic format that can be processed by latex or pdflatex,
% resp., then you can add a logo as follows:

% \pgfdeclareimage[height=0.5cm]{university-logo}{university-logo-filename}
% \logo{\pgfuseimage{university-logo}}

% Delete this, if you do not want the table of contents to pop up at
% the beginning of each subsection:
\AtBeginSubsection[]
{
  \begin{frame}<beamer>{Outline}
    \tableofcontents[currentsection,currentsubsection]
  \end{frame}
}

% Let's get started
\begin{document}

\begin{frame}
  \begin{center}
  \includegraphics [width =0.5\textwidth ]{logo_ugr}
  \end{center}
    \begin{center}
	\includegraphics [width =0.4\textwidth ]{logo_rey}
	\end{center}
  \titlepage


\end{frame}

\begin{frame}{Index}
  \tableofcontents
  % You might wish to add the option [pausesections]
\end{frame}

\section{Introducción, Motivación y objetivos}
%%%%%%%%%%%%%%%%%%%%%%%%%%%%%%%%%%%%%%%%%%%%%%%%%%%%%%%%%%%%%%%%%%%%%%%%%%%%%%%
\begin{frame}{Metodología de trabajo}
Rápidamente sobre que hemos usado git, appear, y el diagrama de Gantt
\end{frame}
\begin{frame}{Objetivos}
Objetivos principales de este trabajo
\end{frame}
\begin{frame}{FPGAs Libres y IceZum Alhambra}
Introducción sobre FPGAs libres y presentación de IceZum con sus carasterísticas breves
\end{frame}
\begin{frame}{IceStudio}
¿Qué es IceStudio y para que nace? Ejemplos de su uso
\end{frame}

%%%%%%%%%%%%%%%%%%%%%%%%%%%%%%%%%%%%%%%%%%%%%%%%%%%%%%%%%%%%%%%%%%%%
\section{Robot Balancín}
\subsection{Diseño del sistema}
\begin{frame}{Diseño del sistema}
A muy alto nivel, el problema planteado y una descripción rápida de la solución propuesta. A partir de aquí, vamos con cada una de las partes
\end{frame}	
\subsection{Implementación del sistema}
\begin{frame}{Estructura mecánica}
Muy breve, porque esa estructura y no otra(DMP) 
\end{frame}
\begin{frame}{Obtención ángulo}
Muy breve como hemos usado la IMU, con un microcontrolador y porque esa IMU y no otra
\end{frame}
\begin{frame}{Coexistencia microcontrolador-FPGA}
Comunicación para mandar angulo de micro a fpga
\end{frame}
\begin{frame}{Control PID}
Se necesita minimizar el error, para ello se usa PID.
Se explica muy breve el PID centrándome mas en las ventajas de hacerlo con la FPGA pues seguramente ellos ya conozcan un PID
\end{frame}
\begin{frame}{Control de los motores}
Driver para motores y como se implementa el generador PWM
No se si seria bueno porque mis diagramas de flujos de cada modulo y explicar alguno
\end{frame}
\begin{frame}{Diseño e implementación PCB}
Hay demasiados cables sueltos y hacemos una PCB, porque 4 capas, porque jumpers, porque posibilidad para 4 motores.
\end{frame}
\subsection{Experimentos y sistema final}
\begin{frame}{Impresión, montaje y ensamblado}
Fotos del ensamblado y vídeo final del sistema.
Debería meter aquí el módulo VGA que hice para aprender y el control de brushless?	
\end{frame}
%%%%%%%%%%%%%%%%%%%%%%%%%%%%%%%%%%%%%%%%%%%%%%%%%%%%%%%%%%%%%%%%%%%%
\section{Cuadricóptero con visión artificial}
\begin{frame}{Diseño del sistema}
Dejar claro que como ha sobrado tiempo, se hace esto para que no piensen que no hemos llegado. Diagrama de bloques general y separación entre percepción y control.
\end{frame}
\subsection{Implementación de la percepción}
\begin{frame}{OV7670 y protocolo I2C}
Porque se ha usado esa cámara, y se dice que se ha implementado un protocolo i2c necesario para los registros, me tire dos meses con ello y tiene que salir :). Se muestra diagrama de bloques del i2c
\end{frame}
\begin{frame}{Reconocimiento del volumen y posición}
Las formulas básicas de como hemos hecho esa percepción y la ventaja de hacer eso con una FPGA, no se necesita memoria externa. 
\end{frame}
\subsection{Diseño del control}
\begin{frame}
Se deja claro que esto falta por implementar pero todo el diseño esta propuesto y debería funcionar. Se explica rápido.
\end{frame}
\section{Conclusiones y trabajo futuro}
\begin{frame}{Conclusiones}
Conclusiones de este trabajo
\end{frame}
\begin{frame}{Trabajo futuro}
Posible trabajo futuro
\end{frame}
\end{document}


