\chapter{Conclusions and future work}\label{sec: Conclusiones}

With this chapter the project documentation is closed. Throughout this report, design and implementation from two robotic systems have been exposed by using open FPGAs.
The key points developed in this project can be summarized ahead:
\begin{itemize}
	\item Learning about educational robotic and open projects
	\item Design and implementation of the mechanical structure of the self-balance robot through the usage of SolidWorks and 3D printing
	\item Design and implementation of a PCB shield for IceZum Alhambra II by using Altium Designer
	\item Hardware implementation using PID controller Verilog and all other systems that compose the self-balance robot
	\item Design and implementation of the quad-copter ball recognition protocol
	\item Low-cost camera reading through FPGA and Verilog
	\item Design of the quad-copter control system
	
\end{itemize}

Through this project, there have been a lot of learned aspects that has nothing specifically to do with the work done. Between them the problem resolution capability and autonomous work can stand out, through which directors have served as a guide, bringing necessary tools for different problems resolution. \newline

The objectives settled at the beginning of the work have been achieved, however, some of them remain as challenges that could not be covered or that have come out during the development of the same. The main challenges that came out for future work are presented below:
\begin{itemize}
	\item Determine a more accurately the mathematical model of the balancing robot for an improvement in the mechanical structure.
	\item Integrate a PID controller (It is a PD controller until now) or use other control systems that could improve the stability.
	\item Improvement in the I2C protocol developed to allow some errors in transmission.
	\item Improvement in the FPGA-Microcontroller communication protocol to raise the speed and allow bidirectional communication. 
	\item Improvement in the recognition of the red ball to allow a less ideal performance environment. 
	\item Implementation of control on board in a quadcopter.
\end{itemize}
Special interest is shown, and is still working on it, about being able to take what has been learned to the classrooms, specially for little kids, allowing an early evolution in a technology whose future is evident and early.
%Electrónica flexible
%%%%%%%%